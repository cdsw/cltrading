\chapter{Discussion}
\label{ch:discussion}

\section{Results Summary}

Excluding Sortino, four factors can be used to comparing the ten agents: final return for each test dataset, MDD for each test dataset, average over benchmark for each test dataset, and training time in timesteps, episode, and seconds. Agents with continuous action space is marked by ``-C'' and discrete ``-D."

Table \ref{tab:agent-ret} shows the testing results based on the final results from different test datasets. Discrete PPO is by far leading in hourly data, while DQN leads when tested on \texttt{MCr}, and for \texttt{MBu}, the most profitable agent is Continuous A2C.

Discrete PPO and DQN's lead is confirmed by looking at the more general fact: average portfolio return over benchmark (Table \ref{tab:comp-over-bm}). In fact, DQN replaces Continuous A2C as the best agent for \texttt{MBu}, and the worst-performing agent is Continuous A2C, especially in hour-level data.

\begin{longtable}[c]{|r|rrrrrrrr|}
\caption{Agent Return Comparison}
\label{tab:agent-ret}\\
\hline
\multicolumn{1}{|c|}{\multirow{2}{*}{\textbf{$\mathcal{D}$}}} & \multicolumn{8}{c|}{\textbf{Return}} \\ \cline{2-9} 
\multicolumn{1}{|c|}{} & \multicolumn{1}{c|}{\textbf{Benchmark}} & \multicolumn{1}{c|}{\textbf{MACD}} & \multicolumn{1}{c|}{\textbf{DQN}} & \multicolumn{1}{c|}{\textbf{A2C-D}} & \multicolumn{1}{c|}{\textbf{A2C-C}} & \multicolumn{1}{c|}{\textbf{PPO-D}} & \multicolumn{1}{c|}{\textbf{PPO-C}} & \multicolumn{1}{c|}{\textbf{SAC}} \\ \hline
\endfirsthead
%
\multicolumn{9}{c}%
{{\bfseries Table \thetable\ continued from previous page}} \\
\endhead
%
\textbf{\texttt{HBu}} & \multicolumn{1}{l|}{0.2643} & \multicolumn{1}{l|}{0.2161} & \multicolumn{1}{r|}{0.1540} & \multicolumn{1}{r|}{0.2568} & \multicolumn{1}{r|}{0.0060} & \multicolumn{1}{r|}{\textbf{1.6982}} & \multicolumn{1}{r|}{0.6357} & 0.3782 \\ \hline
\textbf{\texttt{HBr}} & \multicolumn{1}{l|}{-0.4484} & \multicolumn{1}{l|}{-0.3389} & \multicolumn{1}{r|}{-0.1184} & \multicolumn{1}{r|}{-0.4543} & \multicolumn{1}{r|}{-0.4824} & \multicolumn{1}{r|}{\textbf{2.2585}} & \multicolumn{1}{r|}{0.2589} & -0.1993 \\ \hline
\textbf{\texttt{MCr}} & \multicolumn{1}{l|}{-0.0361} & \multicolumn{1}{l|}{-0.0148} & \multicolumn{1}{r|}{\textbf{0.0296}} & \multicolumn{1}{r|}{-0.0368} & \multicolumn{1}{r|}{-0.0137} & \multicolumn{1}{r|}{0.0090} & \multicolumn{1}{r|}{-0.0265} & -0.0295 \\ \hline
\textbf{\texttt{MBu}} & \multicolumn{1}{l|}{0.0441} & \multicolumn{1}{l|}{0.0364} & \multicolumn{1}{r|}{0.0357} & \multicolumn{1}{r|}{0.0462} & \multicolumn{1}{r|}{\textbf{0.0480}} & \multicolumn{1}{r|}{0.0475} & \multicolumn{1}{r|}{0.0416} & 0.0222 \\ \hline
\end{longtable}

\begin{longtable}[c]{|r|rrrrrrr|}
\caption{Return Over Benchmark Comparison}
\label{tab:comp-over-bm}\\
\hline
\multicolumn{1}{|c|}{\multirow{2}{*}{\textbf{$\mathcal{D}$}}} & \multicolumn{7}{c|}{\textbf{$\overline{\Delta\phi_{\alpha,\odot}}$}} \\ \cline{2-8} 
\multicolumn{1}{|c|}{} & \multicolumn{1}{c|}{\textbf{MACD}} & \multicolumn{1}{c|}{\textbf{DQN}} & \multicolumn{1}{c|}{\textbf{A2C-D}} & \multicolumn{1}{c|}{\textbf{A2C-C}} & \multicolumn{1}{c|}{\textbf{PPO-D}} & \multicolumn{1}{c|}{\textbf{PPO-C}} & \multicolumn{1}{c|}{\textbf{SAC}} \\ \hline
\endfirsthead
%
\multicolumn{8}{c}%
{{\bfseries Table \thetable\ continued from previous page}} \\
\endhead
%
\textbf{\texttt{HBu}} & \multicolumn{1}{r|}{0.0620} & \multicolumn{1}{r|}{0.0204} & \multicolumn{1}{r|}{0.0035} & \multicolumn{1}{r|}{-0.0602} & \multicolumn{1}{r|}{\textbf{0.5988}} & \multicolumn{1}{r|}{0.1897} & 0.1097 \\ \hline
\textbf{\texttt{HBr}} & \multicolumn{1}{r|}{0.1363} & \multicolumn{1}{r|}{0.3015} & \multicolumn{1}{r|}{-0.0009} & \multicolumn{1}{r|}{-0.0320} & \multicolumn{1}{r|}{\textbf{1.5396}} & \multicolumn{1}{r|}{0.5571} & 0.1361 \\ \hline
\textbf{\texttt{MCr}} & \multicolumn{1}{r|}{0.0118} & \multicolumn{1}{r|}{\textbf{0.0383}} & \multicolumn{1}{r|}{0.0000} & \multicolumn{1}{r|}{0.0168} & \multicolumn{1}{r|}{0.0287} & \multicolumn{1}{r|}{0.0057} & 0.0030 \\ \hline
\textbf{\texttt{MBu}} & \multicolumn{1}{r|}{-0.0008} & \multicolumn{1}{r|}{\textbf{0.0102}} & \multicolumn{1}{r|}{0.0000} & \multicolumn{1}{r|}{0.0023} & \multicolumn{1}{r|}{0.0006} & \multicolumn{1}{r|}{0.0017} & -0.0065 \\ \hline
\end{longtable}

The data about MDD aligns with all other findings except for \texttt{HBu}: for this particular case, Continuous PPO produces 1\% less drawdown than Discrete PPO. Overall, in most cases, RL agents tend to perform better in reducing drawdown than the plain buy-and-hold algorithm in minute level.

\begin{longtable}[c]{|l|rrrrrrrr|}
\caption{Agent MDD Comparison}
\label{tab:comp-mdd}\\
\hline
\multicolumn{1}{|c|}{\multirow{2}{*}{\textbf{$\mathcal{D}$}}} & \multicolumn{8}{c|}{\textbf{MDD}} \\ \cline{2-9} 
\multicolumn{1}{|c|}{} & \multicolumn{1}{c|}{\textbf{Benchmark}} & \multicolumn{1}{c|}{\textbf{MACD}} & \multicolumn{1}{c|}{\textbf{DQN}} & \multicolumn{1}{c|}{\textbf{A2C-D}} & \multicolumn{1}{c|}{\textbf{A2C-C}} & \multicolumn{1}{c|}{\textbf{PPO-D}} & \multicolumn{1}{c|}{\textbf{PPO-C}} & \multicolumn{1}{c|}{\textbf{SAC}} \\ \hline
\endfirsthead
%
\multicolumn{9}{c}%
{{\bfseries Table \thetable\ continued from previous page}} \\
\endhead
%
\textbf{\texttt{HBu}} & \multicolumn{1}{l|}{0.1174} & \multicolumn{1}{l|}{0.1159} & \multicolumn{1}{r|}{0.1812} & \multicolumn{1}{r|}{0.1425} & \multicolumn{1}{r|}{0.1622} & \multicolumn{1}{r|}{0.0725} & \multicolumn{1}{r|}{\textbf{0.0718}} & 0.0879 \\ \hline
\textbf{\texttt{HBr}} & \multicolumn{1}{l|}{0.5199} & \multicolumn{1}{l|}{0.3570} & \multicolumn{1}{r|}{0.1343} & \multicolumn{1}{r|}{0.5174} & \multicolumn{1}{r|}{0.5618} & \multicolumn{1}{r|}{\textbf{0.1014}} & \multicolumn{1}{r|}{0.1766} & 0.2915 \\ \hline
\textbf{\texttt{MCr}} & \multicolumn{1}{l|}{0.0725} & \multicolumn{1}{l|}{0.0427} & \multicolumn{1}{r|}{\textbf{0.0313}} & \multicolumn{1}{r|}{0.0719} & \multicolumn{1}{r|}{0.0462} & \multicolumn{1}{r|}{0.0363} & \multicolumn{1}{r|}{0.0551} & 0.0439 \\ \hline
\textbf{\texttt{MBu}} & \multicolumn{1}{l|}{0.0164} & \multicolumn{1}{l|}{0.0124} & \multicolumn{1}{r|}{\textbf{0.0061}} & \multicolumn{1}{r|}{0.0165} & \multicolumn{1}{r|}{0.0101} & \multicolumn{1}{r|}{0.0146} & \multicolumn{1}{r|}{0.0118} & 0.0105 \\ \hline
\end{longtable}

Lastly, it is undoubtably true that the buy-and-hold and MACD agents win the training time comparison, since they do not require any training to operate. Among RL algorithms, Discrete A2C only takes around 4,500 timesteps (4 episodes, 90 seconds) to find its ideal model, yet at the end it just mimics the buy-and-hold agent's performance. The second fastest model is DQN, with clear advantage when tested with minute-level test datasets. Continuous PPO converges before the episode limit (200 episodes), and results in a decent lead over benchmark. Discrete PPO and SAC failed to converge before the training episode limit (150 and 100, respecively), while it is difficult to tell if Continuous A2C would ever converge, due to its unstable nature.

\begin{longtable}[c]{|r|rrrrrr|}
\caption{Time Performance Comparison}
\label{tab:comp-time}\\
\hline
\multicolumn{1}{|c|}{\multirow{2}{*}{\textbf{}}} & \multicolumn{6}{c|}{\textit{\textbf{s}}} \\ \cline{2-7} 
\multicolumn{1}{|c|}{} & \multicolumn{1}{c|}{\textbf{DQN}} & \multicolumn{1}{c|}{\textbf{A2C-D}} & \multicolumn{1}{c|}{\textbf{A2C-C}} & \multicolumn{1}{c|}{\textbf{PPO-D}} & \multicolumn{1}{c|}{\textbf{PPO-C}} & \multicolumn{1}{c|}{\textbf{SAC}} \\ \hline
\endfirsthead
%
\multicolumn{7}{c}%
{{\bfseries Table \thetable\ continued from previous page}} \\
\endhead
%
\textbf{Timesteps} & \multicolumn{1}{r|}{768,936} & \multicolumn{1}{r|}{\textbf{$\sim$45,000}} & \multicolumn{1}{r|}{> 1,671,600} & \multicolumn{1}{r|}{> 2,228,800} & \multicolumn{1}{r|}{$\sim$1,800,000} & > 1,114,400 \\ \hline
\textbf{Episodes} & \multicolumn{1}{r|}{69} & \multicolumn{1}{r|}{\textbf{4}} & \multicolumn{1}{r|}{> 150} & \multicolumn{1}{r|}{> 200} & \multicolumn{1}{r|}{$\sim$167} & > 100 \\ \hline
\textbf{Time} & \multicolumn{1}{r|}{$\sim$2,500} & \multicolumn{1}{r|}{\textbf{$\sim$90}} & \multicolumn{1}{r|}{> 5,500} & \multicolumn{1}{r|}{> 5,000} & \multicolumn{1}{r|}{$\sim$5,100} & > 25,800 \\ \hline
\end{longtable}

\section{Findings}
(to be completed)

\section{Shortcomings}
(to be completed)

\section{Recommendations}
(to be completed)