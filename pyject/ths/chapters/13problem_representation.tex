\chapter{Requirements and Analysis}
\label{Requirement and Analysis}
% no \IEEEPARstart
This chapter provides more detailed insights on the requirements and the flow of the system using activity diagrams. Requirements are represented in a form of FURPS+ model (Table~\ref{requirements table}) which can aid in discovering potential needs that are both functional and non-functional. As the system contains two major subsystems communicating through Wi-Fi protocol, some activity diagrams contain usages of signals to emphasize that some actions are not instantaneous in the system.

\section{Requirements}
\begin{center}
\begin{longtable} {|p{0.03\linewidth}|p{0.72\linewidth}|p{0.16\linewidth}|}
%\renewcommand{\arraystretch}{1.3}
%\caption[Feasible triples for a highly variable Grid]{Feasible triples for 
%highly variable Grid, MLMMH.} \label{grid_mlmmh} \\
\caption{Table of Requirements}
\label{requirements table} \\
%\centering

\hline \multicolumn{1}{|c|}{\textbf{No}} & \multicolumn{1}{c|}{\textbf{Requirement}} & \multicolumn{1}{c|}{\textbf{Type}} \\ \hline 
\endfirsthead

\multicolumn{3}{c}%
{{\bfseries \tablename\ \thetable{} -- continued from the previous page}} \\
\hline \multicolumn{1}{|c|}{\textbf{No}} &
\multicolumn{1}{c|}{\textbf{Requirement}} &
\multicolumn{1}{c|}{\textbf{Type}} \\ \hline 
\endhead

%\hline \multicolumn{3}{|r|}{{Continued on next page}} \\ \hline
%\endfoot

%\hline \hline
%\endlastfoot

%\begin{tabular}{|p{0.03\linewidth}|p{0.8\linewidth}|p{0.1\linewidth}|}
\hline
%\bfseries No & \bfseries Requirement & \bfseries Type\\
%\hline\hline
1 & The luggage shall be able to follow the user. & Functional\\	\hline
2 & User shall be able to use multiple luggage at once. & Functional\\	\hline
3 & User shall be able to manually control the movement of the luggage via a mobile application. & Functional\\	\hline
4 & User shall be able to control the preferable distance between the user and the luggage. & Functional\\	\hline
5 & The luggage shall be able to calculate the relative direction between the luggage and the user. & Functional\\	\hline
6 & The luggage shall be able to calculate user's position relative to the luggage. & Functional\\	\hline
7 & The luggage shall notify the user when the user is out of reach. & Functional\\	\hline
8 & User shall be able to control the luggage via smartphone. & Functional\\	\hline
9 & User shall be able to start and stop the luggage from following. & Functional\\	\hline
10 & User shall be able to manually set off the alarm on the luggage. & Functional\\	\hline
11 & User shall not have to carry any peripheral device except for a smartphone. & Usability\\	\hline
12 & The system shall persist while the relative distance is less than 3 meters. & Reliability\\	\hline
13 & The luggage shall reconnect to the user's smartphone within 3 seconds after it reenters the reachable range. & Reliability\\	\hline
14 & The luggage shall be able to calculate relative position in free space with the tolerance of $ \pm $0.3 meters. & Performance\\	\hline
15 & The system shall support at least two luggage at once. & Performance\\	\hline
16 & The luggage shall support one user at a time. & Performance\\	\hline
17 & The system shall be compatible Android 4.1 or later. & Supportability\\	\hline
18 & The luggage shall not connect to other smartphones while the current connection is not terminated. & Security\\	\hline
%\end{tabular}
\end{longtable}
\end{center}

\section{Use Case Diagram}
\begin{figure}[H]
\centering
\includegraphics{graphics/usecase}
\caption{Use Case diagram}
\label{fig:usecase}
\end{figure}
\pagebreak
\section{Activity Diagrams}
\begin{figure}[H]
\centering
\includegraphics[width={0.55\textwidth}]{graphics/activity_01}
\caption{Connecting to luggage}
\label{fig:activity_01}
\end{figure}
\begin{figure}[H]
\centering
\includegraphics[width={0.55\textwidth}]{graphics/activity_02}
\caption{Disconnecting from luggage}
\label{fig:activity_02}
\end{figure}
\pagebreak
\begin{figure}[H]
\centering
\includegraphics[width={0.8\textwidth}]{graphics/activity_03}
\caption{Activating following mode}
\label{fig:activity_03}
\end{figure}
\begin{figure}[H]
\centering
\includegraphics[width={0.8\textwidth}]{graphics/activity_04}
\caption{Deactivating following mode}
\label{fig:activity_04}
\end{figure}
\pagebreak
\begin{figure}[H]
\centering
\includegraphics[width={0.8\textwidth}]{graphics/activity_05}
\caption{Setting preferable distance}
\label{fig:activity_05}
\end{figure}
\begin{figure}[H]
\centering
\includegraphics[width={0.8\textwidth}]{graphics/activity_06}
\caption{Calculating relative distance}
\label{fig:activity_06}
\end{figure}
\begin{figure}[H]
\centering
\includegraphics[width={0.63\textwidth}]{graphics/activity_07}
\caption{Determining following direction}
\label{fig:activity_07}
\end{figure}
\begin{figure}[H]
\centering
\includegraphics[width={0.63\textwidth}]{graphics/activity_08}
\caption{Controlling movement}
\label{fig:activity_08}
\end{figure}
\begin{figure}[H]
\centering
\includegraphics[width={0.8\textwidth}]{graphics/activity_09}
\caption{Activating alarm}
\label{fig:activity_09}
\end{figure}
\begin{figure}[H]
\centering
\includegraphics[width={0.8\textwidth}]{graphics/activity_10}
\caption{Deactivating alarm}
\label{fig:activity_10}
\end{figure}